\documentclass[a4paper,12pt]{report}
\input{dimensions_a4}

\def\DocLanguage{en}
\input{packages}
\usepackage{tipa}

\def\title{English Grammar}
\def\subtitle{}
\def\author{Havránek Kryštof}

\input{macros}
\input{doc_notes}

\begin{document}

\titlep
\tocp


\chapter{Grammar}

\section{Overview of English Tenses}

\textbf{Present Simple}


\begin{itemize}
	\item Repeated/regular action in the present
	\item General validity: He likes food.
	\item Sequential actions: It's wet after it rains.
	\item Timetabled/scheduled future actions: Train departs at six.
\end{itemize}

\noindent
\textbf{Present Continuous}

\begin{itemize}
	\item Actions currently in progress: He is jumping the jumping jack.
	\item Temporary situations:
	\item Future plans and arrangements: I'm meeting the CEO this evening.
\end{itemize}

\noindent
\textbf{Past Simple, Preterite}

\begin{itemize}
	\item Completed past actions (one-off or repeated): He died in 16th century.
	\item Sequential past actions: After I ate, I went to sleep.
\end{itemize}

\noindent
\textbf{Past Continuous}

\begin{itemize}
	\item Actions already in progress at a specific moment in the past: He was speaking at conference.
	\item Multiple actions in progress at the same time: While he was speaking it was raining outside.
	\item Background description in a narrative
\end{itemize}

\noindent
\textbf{Present Perfect Simple}

\begin{itemize}
	\item Completed past action without a concrete time marker: I have visited that Museum.
	\item Past action with an influence on the present: It seems I have lost my keys.
	\item Action that lasts to the present moment: I have known her for the last 10 years.
	\item Recently completed actions: I've just washed the car.
	\item How much/how many times an action happened up to now: I have seen that movie five times.
\end{itemize}

\noindent
\textbf{Present Perfect Continuous}

\begin{itemize}
	\item How long an action has been in progress up to now: He has been hiking for the last five days.
	\item Recently completed actions with an emphasis on the duration: Why are your clothes dirty? — I have been gardening.
\end{itemize}

\noindent
\textbf{Past Perfect Simple}

\begin{itemize}
	\item Actions that occurred prior to another point in the past: Caesar had had spoken with senators prior to being stabbed.
	\item Sometimes interchangeable with past perfect continuous
\end{itemize}

\noindent
\textbf{Past Perfect Continuous}

\begin{itemize}
	\item Action before a certain point in the past: His hands were dirty because he had been fixing the car.
	\item Emphasizes the action or length of the action: By the time the guests arrived, I had been cooking for two hours.
	\item Sometimes interchangeable with past perfect simple (just emphasizes the duration of the action in the past)
\end{itemize}

\noindent
\textbf{Future denoted with will}

\begin{itemize}
	\item Spontaneous decisions: I think I will eat a banana.
	\item Promises: I will do regular exercise next year.
	\item Predictions: I will rain.
\end{itemize}

\noindent
\textbf{Future denoted with going to}

\begin{itemize}
	\item Intentions for the future: I'm going to change the world for the better.
	\item Logical conclusions regarding the future: War in Ukraine is going to deteriorate global security.
\end{itemize}

\noindent
\textbf{Future Continuous}

\begin{itemize}
	\item Actions that will already be in progress at a certain point in the future: In one year I will be swimming in cash.
\end{itemize}

\noindent
\textbf{Future Perfect Simple}

\begin{itemize}
	\item Actions that will have been completed by a future time: By monday I will have checked with all praetorians whether to proceed with the plan.
\end{itemize}

\noindent
\textbf{Future Perfect Continuous}

\begin{itemize}
	\item Actions that will already have been completed by a future time: He will have been speaking.
\end{itemize}


\section{Reported Speech}

\section{Conditionals}

\subsection{0th Conditional}

\subsection{1st Conditional}

\subsection{2nd Conditional}

\subsection{3rd Conditional}

\subsection{Mixed Conditional}



\section{Unreal Time and Subjunctives}

\section{Phrasal Verbs}

\section{Writing Commas}

\chapter{Spoken Language}

\section{Stress}

TODO: English Pronunciation in Use Advanced by Martin Hewings, suffixes and word stress (Strong and weak folder)

\section{Weak forms of function words}

In spoken English we emphasis is often placed on certain words in a sentence.
Some words aren't usually emphasized. Examples of these include things like articles, indefinite pronouns, prepositions, conjunctions and so on.
In some words that can be, but don't have to be, emphasized there is a difference in pronunciation.

We usually use strong form for one of three reasons:
\begin{enumerate}
	\item Word forms a sentence on it's own.
	\item We want word to be prominent.
	\item Word is used at the end of the sentence.
\end{enumerate}

\begin{center}
	\begin{tabular}{| m{4cm} | m{5cm} || m{3cm} | m{3cm} | }
		\hline
		Mechanism & Words & Weak & Strong \\
		\hline
		Weak form with /\textschwa/ & the, a, an, and, but, that, than, your, then, us, at, for, from, of, to, as, there, can, could, shall, should, would, must, do, does, am, are, was, were, some & I can (/k\textschwa n/) draw \newline This is for (/f\textschwa/) YOU & I CAN (/k\ae n/) come after ALL \newline Who's it FOR (/f\textopeno\textlengthmark/)\\
		\hline
		Weak form with reduced vowels & she, he, we, you & Are you (/ju/, /j\textschwa/) TIRED?  & A: who DID it? \newline B: he (/hi\textlengthmark/) \\
		\hline
		Weak form without (/h/) sound & his, her, he, him, had, had & was he (/hi/, /i/) THERE? & HE (/hi\textlengthmark/) was there but SHE (/\textesh i \textlengthmark/) wasn't \\
		\hline
	\end{tabular}
\end{center}

TODO: add prominent function words (Strong and weak folder)


\chapter{Random tidbits}

\section{Special phrases}

\subsection{Have something done}

We use structure \textbf{have something done} to denote that a action with given object has been arrange to be done by somebody else.
E.g.: Peter repaired the car. (Peter himself repaired the car), Peter had the car repaired. (He arranged a mechanic to repaired the car).
In sentence the order of words is fixed - \textbf{have} - \textbf{subject} - \textbf{past participle}.
There is also option to use the word \textbf{get} instead of \textbf{have}. This is used essentially exclusively in spoken English and isn't particularly formal.

Sometimes structure have something done might not necessary mean that the action was arranged, but that something happened to somebody/somebody's belonging.
E.g.: Peter had his nose broken. Have you ever had your phone stolen.






\section{Similar words}



\subsection{Through, Tough, Thorough, Thought}


\begin{center}
  \begin{tabular}{| m{3cm} || m{12cm} | }
    \hline
    Word     & Meaning                                                                                                     \\
    \hline
    \hline
    Though   & in spite of, however, nevertheless                                                                          \\
    \hline
    Thought  & an individual act or product of thinking                                                                    \\
    \hline
    Tough    & difficult to accomplish, capable of enduring strain                                                         \\
    \hline
    Through  & preposition/adverb: moving from one side to another \newline adverb: continue in time toward the completion \\
    \hline
    Thorough & complete in all respects                                                                                    \\
    \hline
  \end{tabular}
\end{center}

\subsubsection{Examples}

\begin{itemize}
  \item Though:
    \begin{itemize}
      \item Conjunction: I am eagerly awaiting the arrival of spring, though I truly enjoy the winter.
      \item Adverb: The students can be loud and boisterous, but I enjoy working with them, though.
    \end{itemize}
  \item Thought:
    \begin{itemize}
      \item Noun: His thought was that the movie did no justice to the book.
      \item Verb: While daydreaming in class, Morgan often thought about her upcoming graduation. (past tense of think)
    \end{itemize}
  \item Tough:
    \begin{itemize}
      \item The tough new laws are meant to deter tobacco companies from advertising to a young demographic. The tourists were advised to avoid the tough parts of town.
    \end{itemize}
  \item Through:
    \begin{itemize}
      \item Preposition: The kitten snuck in through the garage to stay warm.
      \item Adverb: When we opened the door to the library, the kids came running through. (movement from place to place)
      \item Adverb: The students will be training all summer break, from June through August. (continuation in time towards completion)
      \item Adjective: The employees took a break once they were through with all their tasks. (acts like finished)
    \end{itemize}
  \item Thorough:
    \begin{itemize}
      \item Adjective: After a thorough investigation, the police department concluded that the student was not involved with the recent string of laptop thefts.
    \end{itemize}
\end{itemize}

\subsection{Lie, Lie, Lay}

\begin{itemize}
  \item Lay (verb): place down
  \item Lie (verb): to be in horizontal position, to tell a lie
  \item Lie (noun): untruth
\end{itemize}

\begin{center}
  \begin{tabular}{| m{3.5cm} || m{3.5cm} | m{3.5cm} | m{3.5cm} | }
    \hline
    Tense              & Lay    & Lie (horizontal) & Lie (tell a lie) \\
    \hline
    \hline
    present            & lay    & lie              & lie              \\
    \hline
    past               & laid   & lay              & lied             \\
    \hline
    past participle    & laid   & lain             & lied             \\
    \hline
    present participle & laying & lying            & lying            \\
    \hline
  \end{tabular}

\end{center}


\end{document}
